\chapter{Huffmann}

\section{Rappels sur les probabilités}
On dit qu'une expérience est aléatoire si on peut pas prédire son rédultat. \\
Une \emph{issue} est le résultat d'une expérience aléatoire. \\
L'\emph{univers} $\Omega$ est l'ensemble de toutes ses issues

\paragraph{Exemple}
L'univers de l'expérience aléatoire consitant à lancer une dois un dé à six face est

\[
  \Omega = \{ 1, 2, 3, 4, 5, 6 \}
\]

\begin{tabular}{| c || c c c c |}
  \hline
  $s$    & PP   & PF   & FP   & FF   \\
  \hline
  $p(s)$ & 0.25 & 0.25 & 0.25 & 0.25 \\
  \hline
  $I(s)$ & 2    & 2    & 2    & 2    \\
  \hline
  Code   & 00   & 01   & 10   & 11   \\
  \hline
  $l(s)$ & 2    & 2    & 2    & 2    \\
  \hline
\end{tabular}

\subsection{Exercice 2.3.4}

